
\section{Introduction}
There have been multiple implementation of meta programming in the ML family of languages. Notable implementations are \cite{metaML} and \cite{metaOCaml} that both add two-level annotations with escapes, \texttt{\~}, and templates, \texttt{<>} to Standard ML and OCaml respectively. Brackets, or templates, \texttt{<>} enclose code to be generated and escapes \texttt{\~} mark a hole in the template to insert code \cite{metaOCaml}. One of the notable features of MetaML and Meta-OCaml is that the languages enables the programmer write type checked program generators \cite{metaOCaml}.

\cite{multilevel} on the other hand introduce a novel multi-level specialization
to a lisp like language. Glück and Jørgensen introduce the notion of multi-level generating extensions, where program generation of an \texttt{n} input program can be staged into \texttt{n} stages, with the result of the \texttt{n-1} stages producing a new program generator. They also provide a novel binding time analysis, that given the binding time of the program input, finds an optimal multi-level annotation of the \textit{MetaScheme} language.\\

% TODO: Make it clear that type safety is the motivation
The goal of this work is to combine the work of multi-level program generation described by \cite{multilevel} to a subset of \textit{OCaml} while preserving the type safe guarantees of OCaml similar to \cite{metaOCaml}. In other words, the introduction of multi-level generating extensions to the language should not make well typed programs go wrong if the equivalently non-annotated source program would not go wrong. This will further be discussed in a later section. The extended subset of OCaml is called \textit{mocaml}. %TODO: Make sure you talk about this. 
The focus of this study is not in binding time analysis, but instead handwriting program generators in a multi-level language. Therefore, the multi-level constructs of mocaml must be manually annotated with binding time information. Additionally, the arguments of a mocaml program is assumed to have binding time $1,2,...,n$ in the order they are listed for an $n$ input program.

% TODO explain target language is ocaml with the multi level programs written in mocaml
% TODO a section about generating extensions will also be needed.

%%% Local Variables:
%%% mode: LaTeX
%%% TeX-master: "mocaml"
%%% End:
