\section{Introduction}
There have been multiple implementations of meta programming in the ML family of languages. Notable implementations are \cite{metaML} and \cite{metaOCaml} that both add two-level annotations with escapes, \texttt{\~}, and templates, \texttt{<>} to Standard ML and OCaml respectively. Brackets or templates, \texttt{<>}, enclose code to be generated and escapes \texttt{\~} mark a hole in the template to insert code \cite{metaOCaml}. One of the notable features of MetaML and Meta-OCaml is that the languages enables the programmer write type checked program generators \cite{metaOCaml}.

\cite{multilevel} on the other hand introduces a novel multi-level specialization
to a lisp like language. Glück and Jørgensen introduce the notion of multi-level generating extensions, where program generation of an \texttt{n} input program can be staged into \texttt{n} stages, with the result of the \texttt{n-1} stages producing a multi-level generating extension. They also provide a novel binding time analysis (BTA) that given the binding time of the program input finds an optimal multi-level annotation of the \textit{MetaScheme} language.\\

The goal of this work is to enable the user to hand-write program generators. This is done by combining multi-level annotations similar to those in \cite{multilevel} to a subset of \textit{OCaml} while preserving the type safe guarantees of OCaml similar to \cite{metaOCaml} and \cite{metaML}. In other words, the introduction of multi-level generating extensions to the language should not make well typed programs go wrong if the equivalently non-annotated source program would not go wrong. This will further be discussed in a later section. The extended subset of OCaml is called \textit{mocaml}. The target language after specialization is still OCaml and the program generator is also written in OCaml.

The focus of this study is not in binding time analysis, but instead handwriting program generators in a multi-level language. Therefore, the multi-level constructs of mocaml must be manually annotated with binding time information. Additionally, the arguments of a mocaml program is assumed to have binding time $1,2,...,n$ in the order they are listed for an $n$ input program.

%%% Local Variables:
%%% mode: LaTeX
%%% TeX-master: "mocaml"
%%% End:
