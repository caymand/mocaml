
\section{mocaml Semantics}

\subsection{Type Chcking}

OCaml is a strongly typed language with the property that well-typed OCaml programs do not go wrong as described in \cite{MilnerRobin1978Atot}. This is one of the main motivation for adding multi-level annotations to OCaml - handwritten program generators are still well-typed. This however only dictates that the OCaml types are consistent. It must also be ensured that the binding-time annotations of a mocaml program is consistent. Both aspect of well-typed mocaml programs will be discussed in this section.

\subsubsection{Binding Time}

Since there is no multi-level binding time analysis for mocaml programs, it is up to the programmer to add the correct binding time value to each multi-level annotation. During

\subsection{Call Unfolding}
% TODO: Explain what unfolding is and find reference
% TODO: Also explain why we allow infinite code unfolding, and what alternatives that are provided.
Using recursion in mocaml can cause problems with infinite call unfolding during specialization. For the multiplication program shown in figure \ref{fig:ml_mul}, the specialization simply unfolded the recursive call $n=3$ times. This succeeded because the condition has binding time $t=1$ and the specializer can thereby detect when the branch eventually becomes \texttt{true} and recursion stops. However, if the branch had instead conditioned on $m < 1$, then the binding time would be $t = 2$, and the \texttt{if-then-else} expression would be dynamic at the current stage. As a consequence, if the specializer tries to unroll the recursion, it will run into infinite recursion. 


%%% Local Variables:
%%% mode: LaTeX
%%% TeX-master: "mocaml"
%%% End:
